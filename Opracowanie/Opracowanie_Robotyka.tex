\documentclass{article}
\usepackage{polski}
\usepackage[utf8]{inputenc}
\usepackage{natbib}
\usepackage{graphicx}
\usepackage{xcolor}
\usepackage{mathtools}
\usepackage[makeroom]{cancel}
\usepackage{hyperref}
\newcommand{\norm}[1]{\left\lVert#1\right\rVert}

\title{Opracowanie Robotyka}
\author{Jan Bronicki}
\date{}


\begin{document}

\maketitle



\section{Wykład 1}


\subsection{Warunki, aby macierz była macierzą obrotu}

Aby macierz obrotu {\bf R} była macierzą obrotu musi spełniać warunki:

\begin{itemize}
    \item $R^{T}R=I_{3}$ - macierz musi być ortogonalna
    \item $det R=+1$ - macierz musi być {\bf \it prawoskrętna}
    \item $T$ - (Translacja) - może być dowolna
\end{itemize}

\subsection{Jak sprawdzić prawoskrętność?}

Dla:

\begin{itemize}
    \item $i$ - wersor osi {\bf X}
    \item $j$ - wersor osi {\bf Y}
    \item $k$ - wersor osi {\bf Z}
\end{itemize}

Warunki prawoskrętności:

\begin{itemize}
    \item $i \times j = k$
    \item $j \times k = i$
    \item $k \times j = j$
\end{itemize}

{\bf $T$} - (Translacja, czyli przesunięcie) - dowolna.

\subsection{Obroty stanowią grupę}

Obroty stanowią grupę, która jest zbiorem działań, w którym jest zdefiniowany element odwrotny i neutralny:

\begin{itemize}
    \item Element neutralny - $I_{3}$
    \item Element odwrotny - $R^{T}$
\end{itemize}

\newpage

\subsection{Macierze obrotów ZXY}

\Large
$$
    rot \left(z, \alpha\right) = \begin{bmatrix}
        {\color{purple} c_{\alpha}} & {\color{red}-}{\color{blue} s_{\alpha}} & 0 \\[0.3em]
        {\color{blue} s_{\alpha}}   & {\color{purple} c_{\alpha}}             & 0 \\[0.3em]
        0                           & 0                                       & 1
    \end{bmatrix}
$$

$$
    rot \left(x, \beta\right) = \begin{bmatrix}
        1 & 0                          & 0                                      \\[0.3em]
        0 & {\color{purple} c_{\beta}} & {\color{red}-}{\color{blue} s_{\beta}} \\[0.3em]
        0 & {\color{blue} s_{\beta}}   & {\color{purple} c_{\beta}}
    \end{bmatrix}
$$


$$
    rot \left(y, \gamma\right) = \begin{bmatrix}
        {\color{purple} c_{\gamma}}             & 0 & {\color{blue} s_{\gamma}}   \\[0.3em]
        0                                       & 1 & 0                           \\[0.3em]
        {\color{red}-}{\color{blue} s_{\gamma}} & 0 & {\color{purple} c_{\gamma}}
    \end{bmatrix}
$$
\normalsize

\section{Wykład 2}

\subsection{Współrzędne jednorodne, rozszerzenie do 4 wymiarów}

\Large
$$
    p_{0} = R_{0}^{1} \cdot p_{1}
$$
\normalsize

\begin{itemize}
    \item $p_{0}$ - położenie wektora w układzie "0"
    \item $p_{1}$ - położenie wektora w ukłądzie "1"
    \item $R_{0}^{1}$ - macierz obrotu z położenia "0" do "1"
\end{itemize}

Uwzględniając Translacje:

\Large
$$
    p_{0} = R_{0}^{1} \cdot p_{1} + T_{0}^{1}
$$
\normalsize

$T_{0}^{1}$ - to inaczej odległość pomiędzy początkami układów współrzędnych

Dodajemy czwarty wymiar, aby łatwiej manipulować:
\Large
$$
    \begin{pmatrix}
        p_{0} \\
        1
    \end{pmatrix}
    =
    \begin{bmatrix}
        \begin{array}{c|c}
            R_{0_{3x3}}^{1} & T_{0_{3x1}}^{1} \\[0.3em]
            \hline
            0_{_{1x3}}      & 1
        \end{array}
    \end{bmatrix}
    \begin{pmatrix}
        p_{1} \\
        1
    \end{pmatrix}
$$
\normalsize

\newpage

Przykłady, na macierzach jednorodnych (gdzie $T = \left[0, 0, 0\right]^{T}$):



\Large
$$
    Rot(x, \alpha) =
    \begin{bmatrix}
        \begin{array}{ccc|c}
            1 & 0                           & 0                                       & 0 \\[0.3em]
            0 & {\color{purple} c_{\alpha}} & {\color{red}-}{\color{blue} s_{\alpha}} & 0 \\[0.3em]
            0 & {\color{blue} s_{\alpha}}   & {\color{purple} c_{\alpha}}             & 0 \\[0.3em]
            \hline
            0 & 0                           & 0                                       & 1
        \end{array}
    \end{bmatrix}
$$
\normalsize


\Large
$$
    Rot(y, \beta) =
    \begin{bmatrix}
        \begin{array}{ccc|c}
            {\color{purple} c_{\gamma}}             & 0 & {\color{blue} s_{\gamma}}   & 0 \\[0.3em]
            0                                       & 1 & 0                           & 0 \\[0.3em]
            {\color{red}-}{\color{blue} s_{\gamma}} & 0 & {\color{purple} c_{\gamma}} & 0 \\[0.3em]
            \hline
            0                                       & 0 & 0                           & 1
        \end{array}
    \end{bmatrix}
$$
\normalsize



\Large
$$
    Rot(z, \gamma) =
    \begin{bmatrix}
        \begin{array}{ccc|c}
            {\color{purple} c_{\alpha}} & {\color{red}-}{\color{blue} s_{\alpha}} & 0 & 0 \\[0.3em]
            {\color{blue} s_{\alpha}}   & {\color{purple} c_{\alpha}}             & 0 & 0 \\[0.3em]
            0                           & 0                                       & 1 & 0 \\[0.3em]
            \hline
            0                           & 0                                       & 0 & 1
        \end{array}
    \end{bmatrix}
$$
\normalsize


\newpage

Przykłady Translacji po danych osiach (bez rotacji):


\Large
$$
    Trans(x, a) =
    \begin{bmatrix}
        \begin{array}{ccc|c}
            1 & 0 & 0 & a \\[0.3em]
            0 & 1 & 0 & 0 \\[0.3em]
            0 & 0 & 1 & 0 \\[0.3em]
            \hline
            0 & 0 & 0 & 1
        \end{array}
    \end{bmatrix}
$$
\normalsize


\Large
$$
    Trans(y, b) =
    \begin{bmatrix}
        \begin{array}{ccc|c}
            1 & 0 & 0 & 0 \\[0.3em]
            0 & 1 & 0 & b \\[0.3em]
            0 & 0 & 1 & 0 \\[0.3em]
            \hline
            0 & 0 & 0 & 1
        \end{array}
    \end{bmatrix}
$$
\normalsize



\Large
$$
    Trans(z, c) =
    \begin{bmatrix}
        \begin{array}{ccc|c}
            1 & 0 & 0 & 0 \\[0.3em]
            0 & 1 & 0 & 0 \\[0.3em]
            0 & 0 & 1 & c \\[0.3em]
            \hline
            0 & 0 & 0 & 1
        \end{array}
    \end{bmatrix}
$$
\normalsize


Inna interpolacja:

Gdy $T_{0}^{1} = 0$ układy nie są przesunięte względem siebie, ale są skręcone:
np (gdzie 1, to i-te miejsce):

\Large
$$
    p_{1}=e_{i}=\begin{pmatrix}
        0 \\
        1 \\
        0
    \end{pmatrix}
$$

$$
    \begin{pmatrix}
        p_{0} \\
        1
    \end{pmatrix}
    =
    \begin{bmatrix}
        \begin{array}{c|c}
            R_{0_{3x3}}^{1} & 0 \\[0.3em]
            \hline
            0_{_{1x3}}      & 1
        \end{array}
    \end{bmatrix}
    \begin{pmatrix}
        e_{1} \\
        1
    \end{pmatrix}
$$
\normalsize

\newpage

\subsection{Dzięki współrzędnym jednorodnym operacja rotacji i translacji jest reprezentowana przez jedną macierz}

Ważne:

\begin{itemize}
    \item Rotacje następujące po sobię są mnożeniem następujących macierzy obrotu
    \item Współrzędne jednorodne są grupą nieprzemienną, z działaniem "mnożenie macierzy"
          $$
              K =
              \begin{bmatrix}
                  R & T \\[0.3em]
                  0 & 1 \\[0.3em]
              \end{bmatrix}
              \implies
              K^{-1} =
              \begin{bmatrix}
                  R^{T} & -R^{T} \cdot T \\[0.3em]
                  0     & 1
              \end{bmatrix}
          $$
\end{itemize}

\subsection{Składanie ruchów}

\Large
\begin{enumerate}
    \item {\bf Względem osi bieżących (np. nasz statek, którym sterujemy)}

          Mnożymy od lewej do prawej:
          $$
              K_{0}^{n} = K_{0}^{\bcancel{1}}\cdot K_{\bcancel{1}}^{\bcancel{2}} \cdot K_{\bcancel{2}}^{3} \cdot \ldots \cdot K_{n-1}^{n}
          $$

          Nie jest ważne przez jakie macierze przechodziliśmy, ważne jest jak skończyliśmy.


          Przykład mnożenie:

          $$
              K_{0}^{1} =
              \begin{bmatrix}
                  R_{0}^{1} & T_{0}^{1} \\[0.3em]
                  0         & 1
              \end{bmatrix}
              , \
              K_{1}^{2} =
              \begin{bmatrix}
                  R_{1}^{2} & T_{1}^{2} \\[0.3em]
                  0         & 1
              \end{bmatrix}
          $$

          $$
              K_{0}^{\bcancel{1}}\cdot K_{\bcancel{1}}^{2} = K_{0}^{2} =
              \begin{bmatrix}
                  R_{0}^{1} & T_{0}^{1} \\[0.3em]
                  0         & 1
              \end{bmatrix}
              \cdot
              \begin{bmatrix}
                  R_{1}^{2} & T_{1}^{2} \\[0.3em]
                  0         & 1
              \end{bmatrix}
              =
              \begin{bmatrix}
                  \begin{array}{c|c}
                      \overbrace{R_{0}^{1}\cdot R_{1}^{2}}^{R_{0}^{2}} & \overbrace{R_{0}^{1}T_{1}^{2}+T_{0}^{1}}^{T_{0}^{2}} \\[0.3em]
                      \hline
                      0                                                & 1
                  \end{array}
              \end{bmatrix}
          $$



    \item {\bf Względem osi ustalonych (np. wybżerza portu)}

          Od prawej do lewej (jest mniej ważne)

\end{enumerate}
\normalsize

\newpage

\subsection{Parametryzjacja obrotów}

\Large
$$
    R_{1}\cdot R_{2} \cdot \ldots \cdot R_{n} = R_{w}
$$
\normalsize

$R_{w}$ - wynik mnożenia macierzy obrotów, także jest obrotem

\subsubsection{Reprezentacja obrotów: Kąty Eulera}

Kąty Eulera (w formie ZYZ):

\Large
$$
    E\left(\alpha, \beta, \gamma\right)=rot\left(z, \alpha\right) \cdot rot\left(y, \beta\right) \cdot rot\left(z, \gamma\right)
$$
\normalsize

Macierz ogólna takiej rotacji:


\Large
$$
    E\left(\alpha, \beta, \gamma\right)=rot\left(z, \alpha\right) \cdot rot\left(y, \beta\right) \cdot rot\left(z, \gamma\right) =
$$
$$
    =
    \begin{bmatrix}
        {\color{purple} c_{\alpha}} & {\color{red}-}{\color{blue} s_{\alpha}} & 0 \\[0.3em]
        {\color{blue} s_{\alpha}}   & {\color{purple} c_{\alpha}}             & 0 \\[0.3em]
        0                           & 0                                       & 1
    \end{bmatrix}
    \cdot
    \begin{bmatrix}
        {\color{purple} c_{\beta}}             & 0 & {\color{blue} s_{\beta}}   \\[0.3em]
        0                                      & 1 & 0                          \\[0.3em]
        {\color{red}-}{\color{blue} s_{\beta}} & 0 & {\color{purple} c_{\beta}}
    \end{bmatrix}
    \cdot
    \begin{bmatrix}
        {\color{purple} c_{\gamma}} & {\color{red}-}{\color{blue} s_{\gamma}} & 0 \\[0.3em]
        {\color{blue} s_{\gamma}}   & {\color{purple} c_{\gamma}}             & 0 \\[0.3em]
        0                           & 0                                       & 1
    \end{bmatrix}
    =
$$
$$
    =
    \begin{bmatrix}
        \begin{array}{c|c|c}
            c_{\alpha}c_{\beta}c_{\gamma}-s_{\alpha}s_{\gamma} & -c_{\alpha}c_{\beta}s_{\gamma}-s_{\alpha}c_{\alpha} & c_{\alpha}s_{\beta} \\[0.3em]
            \hline
            s_{\alpha}c_{\beta}c_{\gamma}+c_{\alpha}s_{\gamma} & -s_{\alpha}c_{\beta}s_{\gamma}+c_{\alpha}c_{\gamma} & s_{\alpha}s_{\beta} \\[0.3em]
            \hline
            -s_{\beta}c_{\gamma}                               & s_{\beta}s_{\gamma}                                 & c_{\beta}
        \end{array}
    \end{bmatrix}
$$
\normalsize

\newpage

\subsubsection{Reprezentacja obrotów: Roll-Pitch-Jaw}

Jest to reprezentacja typu oś-kąt

\Large
$$
    RPY\left(\phi, \theta, \psi\right)=rot(z, \psi) \cdot rot(y, \theta) \cdot rot(x, \psi)
$$
\normalsize


\begin{figure}[h!]
    \centering
    \includegraphics[scale=0.5]{./img/rpy.png}
\end{figure}
\url{https://youtu.be/pQ24NtnaLl8}

\subsubsection{Reprezentacja typu oś-kąt, dla macierzy {\bf R}, k-oś (musi być znormalizowana, czyli $\norm{k}=1$)}

Jak znaleźć oś z macierzy obrotu?

$tr \ R$ - Ślad macierzy {\bf R}

Suma elementów na przekątnej w macierzy {\bf R}:

\Large
$$
    tr \ R=1+2\cdot c_{\theta} \implies \theta=\arccos\left(\frac{tr \ R-1}{2}\right)
    \theta = \arccos\left(\frac{tr \ R - 1}{2}\right)
$$
\normalsize

Obrót w okół osi {\bf k} o kąt $\theta$:

\Large
$$
    \left[k\right]=\frac{R-R^{T}}{2\cdot\sin\theta}=
    \begin{bmatrix}
        0      & -k_{z} & k_{y}  \\[0.3em]
        k_{z}  & 0      & -k_{x} \\[0.3em]
        -k_{y} & k_{x}  & 0
    \end{bmatrix}
    \implies
    k =
    \begin{pmatrix}
        k_{x} \\
        k_{y} \\
        k_{z}
    \end{pmatrix}
$$
\normalsize

\newpage

Operacja odwrotna.
Mamy jakiś wektor {\bf k} i chcemy go unormować, żeby miał długość $1$ i mamy
podany kąt o jaki chcielibyśmy wykonać obrót:


Zwykły wektor $k_{zw}$, który chemy unormować do postaci $k =
    \begin{pmatrix}
        k_{x} \\
        k_{y} \\
        k_{z}
    \end{pmatrix}$, gdzie $\norm{k}=1$

Jakiej macierzy odpowiada obrót wokół wektora {\bf k} o kąt $\theta$?


\Large
$$
    rot(k, \theta)=
    \begin{bmatrix}
        \begin{array}{c|c|c}
            k_{x}^{2}(1-c_{\theta})+c_{\theta}       & k_{x}k_{y}(1-c_{\theta})-k_{z}s_{\theta} & k_{x}k_{z}(1-c_{\theta})-k_{y}s_{\theta} \\[0.3em]
            \hline
            k_{x}k_{y}(1-c_{\theta})+k_{z}s_{\theta} & k_{y}^{2}(1-c_{\theta})+c_{\theta}       & k_{y}k_{z}(1-c_{\theta})-k_{x}s_{\theta} \\[0.3em]
            \hline
            k_{x}k_{z}(1-c_{\theta})-k_{y}s_{\theta} & k_{y}k_{z}(1-c_{\theta})+k_{x}s_{\theta} & k_{z}^{2}(1-c_{\theta})+c_{\theta}
        \end{array}
    \end{bmatrix}
$$
\normalsize

\subsubsection{Obrót układ w okół dowolnego wektora (tutaj $v$) o kąt $\theta$}

Robimy to tak jakbyśmy byli we współrzędnych sferycznych. Na początku jest to rotacja w okół $Z$ o kąt $\alpha$

\Large
$$
    R\left(v, \varphi\right)=
    R(Z, \alpha)R(Y,\beta)R(Z, \varphi)R^{T}(Y, \beta)R^{T}(Z, \alpha)
$$
\normalsize


\begin{figure}[h!]
    \centering
    \includegraphics[scale=0.5]{img/kierunek_osi_obrotu.jpg}
\end{figure}

Ogólną macierz transformacji o dowolny (unormowany) wetkor można policzyć na wiele sposobów.
Jest to transformacja odwrotna do transformacji {\bf oś-kąt}

\newpage

\section{Wykład 3}






\end{document}
